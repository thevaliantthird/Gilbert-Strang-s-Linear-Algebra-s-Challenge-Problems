\documentclass{article}
\usepackage[utf8]{inputenc}
\usepackage{amssymb}
\usepackage{amsmath}
\addtolength{\oddsidemargin}{-.875in}
\addtolength{\evensidemargin}{-.875in}
\addtolength{\textwidth}{1.75in}

\addtolength{\topmargin}{-1in}
\addtolength{\textheight}{1.75in}

\usepackage{graphicx}
\graphicspath{ {./Desktop/Notes/MA106} }

\title{Gilbert Strang's Linear Algebra $5^{th}$ Edition}
\author{Shubh Kumar}
\date{$15^{th}$ of March, 2021}

\begin{document}

\maketitle

\section{Challenge Problems(1.2):}
\begin{itemize}
  \item 30. The Given scenario is very much true in case of vectors in one plane. Consider 3 vectors each pointing to a vertex of an equilateral triangle centered at origin, and that would hold it true, as in case case : $\frac{\mathbf{u \cdot v}}{\|u\| \cdot \|v\|} = \frac{-1}{2}$ and similarly for the other two. \\
  Considering the same thing in 3D-xyz plane would complicate things a bit, but with benefit of hindsight we know the case of Tetrahedron, which'll hold it true for four vectors poiting to its four vertices in which case: For All pair of vectors$(\mathbf{u,v}: \frac{\mathbf{u \cdot v}}{\|u\| \cdot \|v\|} = \cos(109.47) < 0)$
  \\
  Digging Deeper into the question, Back in the 2D case, we find that we won't be able to find 4 or more vectors which hold the condition true as now, Even while considering the symmetric case i.e. The Four Vectors pointing towards the vertices of a square, we'll have $\mathbf{u \cdot v} = 0$, and if we move anyone or more of those vectors, then if the dot product becomes negative for some of them, it'll always become positive for others. \\
  With similar assertions in 3D, we conclude that we can't find any more than 4 such vectors, because if we try to do so in case of 5 or 6 vectors case, then we'll be unable to find any symmetric annalogues which uplhold these conditions.
  \item 31. $x+y+z=0 \implies x^2 + y^2 + z^2 = -2xy -2xz -2yz$\\
  $\implies \frac{xy+yz+xz}{x^2 + y^2 + z^2} = \frac{-1}{2} =  \frac{\mathbf{u \cdot v}}{\|u\| \cdot \|v\|}$ \\ This means that the angle is always $\frac{2\pi}{3}$

  \item 32. The Given Inequality is actually wrong! A Counter Example would be: \\
  $x = \frac{-y}{2}, z = \frac{-y}{2}, y > 0$ \\
  LHS: $\frac{y}{2^{\frac{2}{3}}} > (y-\frac{-y}{2}-\frac{-y}{2}) = 0$

  \item 33. The total 16 possibilites brought forward by $\big(\pm \frac{1}{2}, \pm \frac{1}{2},\pm \frac{1}{2},\pm \frac{1}{2}\big)$ would be lying in the 4-dimensional hyperplane with appropriate symmetry. \\
  Hence, We can choose one arbitarily let it be $\big(\frac{1}{2},\frac{1}{2},\frac{1}{2},\frac{1}{2} \big)$ \\
  Clearly one of the vectors perpendicular to it will be $\big(\frac{1}{2},\frac{-1}{2},\frac{-1}{2},\frac{1}{2} \big)$ \\
  and another one perpendicular to both of them would be $\big(\frac{-1}{2},\frac{-1}{2},\frac{1}{2},\frac{1}{2} \big)$ \\
  and finally one perpendicular to all three of these would be $\big(\frac{1}{2},\frac{1}{2},\frac{-1}{2},\frac{-1}{2} \big)$\\
  Also observe that even if we replace the Second vector with $\big(\frac{-1}{2},\frac{1}{2},\frac{1}{2},\frac{-1}{2} \big)$. The Quadruplet of Vectors would still be mutually-perpendicular to each other.
  \\ To Generalize, we can take any one vector(All +ve in above case), and generate a Quadruplet as:
  $\big(\pm \frac{1}{2}, \pm \frac{1}{2},\pm \frac{1}{2},\pm \frac{1}{2}\big)$ , $\big(\pm \frac{1}{2}, \mp \frac{1}{2},\mp \frac{1}{2},\pm \frac{1}{2}\big)$ or $\big(\mp \frac{1}{2}, \pm \frac{1}{2},\pm \frac{1}{2},\mp \frac{1}{2}\big)$ , $\big(\pm \frac{1}{2}, \pm \frac{1}{2},\mp \frac{1}{2},\mp \frac{1}{2}\big)$ and $\big(\mp \frac{1}{2}, \mp \frac{1}{2},\pm \frac{1}{2},\pm \frac{1}{2}\big)$
\end{itemize}

\section{Challenge Problems(1.2):}
\begin{itemize}
  \item The Matrix C is \\ \\
  $\begin{bmatrix}
    0 && 1 && 0 && 0 && 0 \\
    -1 && 0 && 1 && 0 && 0 \\
    0 && -1 && 0 && 1 && 0 \\
    0 && 0 && -1 && 0 && 1 \\
    0 && 0 && 0 && -1 && 0

  \end{bmatrix}$
\end{itemize}

\end{document}
